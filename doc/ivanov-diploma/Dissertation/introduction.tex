\chapter*{Введение}							% Заголовок
\addcontentsline{toc}{chapter}{Введение}	% Добавляем его в оглавление

С появлением в Интернете социальных медиа ресурсов, таких как социальные сети, форумы, блоги и др. наблюдается рост как количества пользователей этих ресурсов, так и объема данных, создаваемых пользователями. Люди активно делятся своим мнением в комментариях, обзорах, отзывах, обсуждениях. Эта активность представляет огромный практический интерес со стороны бизнеса, поскольку влияет на покупательскую способность \cite{wang}. В частности, одной из актуальных и практически важных задач для бизнеса является анализ тональности \cite{mokoron2012}. 

Анализ тональности текста (документа) является устоявшейся задачей в обработке естественного языка, целью которой является определение полярности ("позитивной", "негативной" или "нейтральной") данного документа. Интерес к данной области стал активно проявляться в последние годы. Существует несколько ежегодных соревнований, таких как, например, соревнование по созданию систем автоматического анализа тональности на английском языке SemEval \cite{semeval2010}, которое проходит с 2010 года.

В задаче анализа тональности широко применяются различные техники обработки естественного языка для извлечения значимых признаков (факторов) и закономерностей из большого набора данных, а также алгоритмы машинного обучения для классификации отдельных примеров текста в соответствии с их признаковым представлением.

При этом признаки могут быть разделены на две группы: основанные на формальном языке и "неформальные". Основанные на формальном языке признаки связаны с формальной лингвистикой и включают априорные знания об эмоциональной окраске (полярности) отдельных слов и фраз. Априорная эмоциональная окраска означает, что некоторые слова и фразы имеют естественную тенденцию к демонстрации определенной тональности. К примеру, слово "хороший" имеет позитивную коннотацию, в то время как слово "плохой" имеет негативную коннотацию. Т.о. если слово с позитивной коннотацией содержиться в предложении, возможно, и предложение в целом имеет позитивную окраску. С другой стороны, определение частей речи - это синтаксический подоход к задаче. Оно заключается в автоматичесом определение к каким частям речи (существительное, прилагательное, глагол и т.д.) относятся каждое отдельное слово из предложения. Закономерности могут быть извлечены при помощи анализа частотного распределения частей речи в определенном классе размеченных данных. Такие признаки являются "неформальными".

Алгоритмы классификации также можно разделить на две большие группы: "с учителем" (supervised) и "без учителя" (unsupervised). Алгоритмы машинного обучения с учителем используют доступные размеченные примеры данных. Обучением классификатора означает использование размеченных данных для извлечения признаков, которые моделируют закономерности и различия между классами, для того чтобы затем классифицировать неразмеченный пример в соответствии с данными признаками. Алгоритмы без учителя, наоборот, не используют разметку.

Для тестирования алгоритмов определения тональности существует также несколько открытых наборов данных. Для английского языка, например, это обзоры фильмов~\footnote{http://nlp.stanford.edu/sentiment/treebank.html}~\cite{imdb, socher-rdm} и товаров~\footnote{http://jmcauley.ucsd.edu/data/amazon/}~\cite{amazon}.
% Наилучшие алгоритмы в данной области, основаны на рекурсивных тензорных нейронных сетях~\cite{socher-rdm}.

Для русского языка в рамках соревнований Семинара РОМИП в 2011 году~\cite{romip} был сформирован корпус текстов, содержащий отзывы о книгах, фильмах и товарах. В 2014 году Рубцова Ю.В. создала корпус коротких сообщений Twitter~\cite{mokoron2015}.
На соревновательной дорожке по анализу тональности Dialogue Evaluate 2016~\cite{senti-ru-eval} был представлен корпус сообщений Twitter, посвященный банкам и телекоммуникационным компаниям. Победителем последнего соревнования в 2016 году стало решение, основанное на рекуррентной нейронной сети~\cite{arhipenko}.

В данной работе рассматривается задача классификации русских текстов по тональности. В качестве классификаторов используются такие модели как двунаправленная рекуррентная нейронная сеть~\cite{schuster} и двунаправленная рекуррентная нейронная сеть с механизмом внимания~\cite{bahdanau}. Целью является сравнение данных моделей. Ранее для анализа тональности русского текста не применялись модели с механизмом внимания, хотя они успели показать впечатляющие результаты в задачах классификации англоязычных текстов~\cite{yang-att-2016}. Представленные в работе классификаторы экспериментально проверяются на нескольких наборах данных.
%указанном выше наборе русских сообщений из Twitter, представленных на Dialogue Evaluate 2016.

Для оценки и сравнения результатов экспериментов используются такие метрики как точность (accuracy) и макро-усредненная F1-мера.
