\thispagestyle{empty}

\vspace{0pt plus1fill} %число перед fill = кратность относительно некоторого расстояния fill, кусками которого заполнены пустые места
\begin{flushright}
  \large{На правах рукописи}
  \includegraphics[height=1.5cm]{personal-signature} 
\end{flushright}

\vspace{0pt plus3fill} %число перед fill = кратность относительно некоторого расстояния fill, кусками которого заполнены пустые места
\begin{center}
\textbf {\large \thesisAuthor}
\end{center}

\vspace{0pt plus3fill} %число перед fill = кратность относительно некоторого расстояния fill, кусками которого заполнены пустые места
\begin{center}
\textbf {\Large \thesisTitle}

\vspace{0pt plus3fill} %число перед fill = кратность относительно некоторого расстояния fill, кусками которого заполнены пустые места
{\large Специальность \thesisSpecialtyNumber\ "---\par <<\thesisSpecialtyTitle>>}

\vspace{0pt plus1.5fill} %число перед fill = кратность относительно некоторого расстояния fill, кусками которого заполнены пустые места
\Large{Автореферат}\par
\large{диссертации на соискание учёной степени\par \thesisDegree}
\end{center}

\vspace{0pt plus4fill} %число перед fill = кратность относительно некоторого расстояния fill, кусками которого заполнены пустые места
\begin{center}
{\large{\thesisCity\ "--- \thesisYear}}
\end{center}

\newpage
% оборотная сторона обложки
\thispagestyle{empty}
\noindent Работа выполнена в \thesisInOrganization

\par\bigskip
%\begin{table}[h] % считается не очень правильным использовать окружение table, не задавая caption
    \noindent%
    \begin{tabular}{@{}lp{0.637\textwidth}}
        \sfs Научный руководитель: & \sfs \supervisorRegalia \par
                                      \textbf{\supervisorFio}
        \vspace{0.013\paperheight} \\
        {\sfs Официальные оппоненты:} &
        {\sfs \textbf{\opponentOneFio,}\par
                  \opponentOneRegalia,\par
                  \opponentOneJobPlace,\par
                  \opponentOneJobPost\par \vspace{0.01\paperheight}
                  \textbf{\opponentTwoFio,}\par \vspace{0.0034\paperheight}
                  \opponentTwoRegalia,\par
                  \opponentTwoJobPlace,\par
                  \opponentTwoJobPost
        }
        \vspace{0.013\paperheight} \\
        {\sfs Ведущая организация:} & {\sfs \leadingOrganizationTitle }
    \end{tabular}  
%\end{table}
\par\bigskip

\noindent Защита состоится \defenseDate~на~заседании диссертационного совета \defenseCouncilNumber~на базе \defenseCouncilTitle~по адресу: \defenseCouncilAddress.

\vspace{0.017\paperheight}
\noindent С диссертацией можно ознакомиться в библиотеке \synopsisLibrary.

\vspace{0.017\paperheight}
\noindent{Автореферат разослан \synopsisDate.}

\vspace{0.017\paperheight}
%\begin{table} [h] % считается не очень правильным использовать окружение table, не задавая caption
\par\bigskip
    \noindent%
    \begin{tabular}{p{0.47\textwidth}cr}
        \begin{tabular}{p{0.34\textwidth}}
            \sfs Ученый секретарь  \\
            \sfs диссертационного совета  \\
            \sfs \defenseCouncilNumber, \defenseSecretaryRegalia
        \end{tabular} 
    &
        \begin{tabular}{c}
            \includegraphics[height=0.034\paperheight]{secretary-signature} 
        \end{tabular} 
    &
        \begin{tabular}{r}
            \\
            \\
            \sfs \defenseSecretaryFio
        \end{tabular} 
    \end{tabular}
%\end{table}
\newpage
